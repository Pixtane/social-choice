\documentclass[14pt]{article}
\usepackage[utf8]{inputenc}
\usepackage[ukrainian]{babel}
\usepackage[T2A]{fontenc}
\usepackage{times}
\usepackage{geometry}
\geometry{a4paper, left=20mm, top=20mm, bottom=20mm, right=10mm}
\usepackage{setspace}
\onehalfspacing
\usepackage{amsmath}
\usepackage{amssymb}
\usepackage{amsthm}
\usepackage{graphicx}
\usepackage{hyperref}
\usepackage{booktabs}
\usepackage{multirow}
\usepackage{array}
\usepackage{natbib}
\usepackage{fontsize}
\renewcommand{\tablename}{Таблиця}
\changefontsize[14pt]{14pt}
\usepackage{caption}
\usepackage{chngcntr}
\usepackage{fancyhdr}
\captionsetup[table]{textformat=simple, labelformat=simple, justification=centering, singlelinecheck=false}
\renewcommand{\figurename}{Рис.}
% Налаштування нумерації таблиць, рисунків та формул за розділами
\counterwithin{figure}{section}
\counterwithin{table}{section}
\counterwithin{equation}{section}
% Налаштування нумерації сторінок (справа зверху, починається з титульної)
\pagestyle{fancy}
\fancyhf{}
\fancyhead[R]{\thepage}
\fancyfoot{}
\renewcommand{\headrulewidth}{0pt}

\newtheorem{theorem}{Теорема}[section]
\newtheorem{lemma}[theorem]{Лема}
\newtheorem{proposition}[theorem]{Твердження}
\newtheorem{definition}[theorem]{Визначення}

\begin{document}

% --- ТИТУЛЬНИЙ АРКУШ ---
\setcounter{page}{1}
\thispagestyle{empty}
\begin{titlepage}
    \centering
    \small
    МІНІСТЕРСТВО ОСВІТИ І НАУКИ УКРАЇНИ \\
    ДЕПАРТАМЕНТ ОСВІТИ І НАУКИ \\
    ВИКОНАВЧОГО ОРГАНУ КИЇВСЬКОЇ МІСЬКОЇ РАДИ \\
    (КИЇВСЬКОЇ МІСЬКОЇ ДЕРЖАВНОЇ АДМІНІСТРАЦІЇ) \\
    КИЇВСЬКЕ ТЕРИТОРІАЛЬНЕ ВІДДІЛЕННЯ МАЛОЇ АКАДЕМІЇ НАУК УКРАЇНИ \\
    (КИЇВСЬКА МАЛА АКАДЕМІЯ НАУК) \par
    \vspace{1cm}
    \normalsize
    Відділення: Математика \\
    Секція: Математичне моделювання \par
    \vspace{3cm}
    \Large
    \textbf{ГЕТЕРОГЕННІСТЬ МЕТРИК У БАГАТОВИМІРНИХ МОДЕЛЯХ СУСПІЛЬНОГО ВИБОРУ} \par
    \vspace{3cm}
    \normalsize
    \flushright
    \textbf{Роботу виконав:} \\
    Карасевич Ярослав Олександрович \\
    повна дата народження \\
    учень 11 класу \\
    Ліцею 315 м. Києва \par
    \vspace{0.5cm}
    \textbf{Науковий керівник:} \\
    Содома Іван Петрович \\
    учитель математики Ліцею 315 \par
    \vfill
    \centering
    КИЇВ -- 2026
\end{titlepage}

\newpage

% --- АНОТАЦІЯ ---
\begin{center}
    \Large
    \textbf{ГЕТЕРОГЕННІСТЬ МЕТРИК У БАГАТОВИМІРНИХ МОДЕЛЯХ СУСПІЛЬНОГО ВИБОРУ}
\end{center}


\noindent \textbf{Автор:} Карасевич Ярослав Олександрович
\textbf{Територіальне відділення:} Київське територіальне відділення МАН України \textbf{Заклад освіти:} Ліцей № 315 м. Києва, 11 клас \textbf{Науковий керівник:} Содома Іван Петрович, учитель математики Ліцею № 315

Робота присвячена математичному моделюванню того, як гетерогенність метрик відстані впливає на агрегування переваг у багатовимірних просторових моделях голосування. Актуальність теми полягає в тому, що більшість формалізованих моделей суспільного вибору припускають одну спільну метрику для всіх виборців, а в багатовимірних просторах це може знижувати чутливість моделі та спотворювати висновки через ефекти концентрації міри.

Метою роботи є побудова й аналіз обчислювальної моделі, яка враховує різні метричні структури під час оцінювання відстаней між виборцями та альтернативами, а також кількісна оцінка того, як ця гетерогенність впливає на результати колективного вибору. У дослідженні поєднано інструменти геометрії високих вимірів і теорії метрик із чисельним моделюванням та симуляціями Монте-Карло. У першому розділі розглянуто поведінку метрик $L_1$, $L_2$ і косинусної подібності у високих вимірах та їхній вплив на структуру відстаней. Другий розділ описує параметризовану модель голосування й алгоритмічну постановку чисельних експериментів. У третьому розділі проаналізовано стійкість результатів агрегування для різних правил соціального вибору.

Моделювання показує, що введення метричної гетерогенності спричиняє систематичні електоральні розбіжності в межах 4–18,5\% навіть за фіксованих просторових конфігурацій. Виявлено, що окремі правила агрегації, зокрема метод Борда, є особливо чутливими до зміни метричних припущень. Отримані результати можуть бути використані для побудови більш стійких і коректних математичних моделей колективного вибору в багатовимірних середовищах.

\textbf{Ключові слова}: просторові моделі голосування, гетерогенність метрик, ідеологічний простір, концентрація міри, метод Борда, симуляція Монте-Карло, електоральна розбіжність, високовимірні дані.

\newpage

% --- ЗМІСТ ---
\tableofcontents

\newpage

% --- ПЕРЕЛІК УМОВНИХ ПОЗНАЧЕНЬ, СИМВОЛІВ, СКОРОЧЕНЬ, ТЕРМІНІВ ---
\section*{ПЕРЕЛІК УМОВНИХ СКОРОЧЕНЬ}
\addcontentsline{toc}{section}{ПЕРЕЛІК УМОВНИХ СКОРОЧЕНЬ}

\begin{tabular}{p{0.3\textwidth}p{0.65\textwidth}}
    $d$                         & Вимірність простору                                                 \\
    $d(\mathbf{x}, \mathbf{y})$ & Відстань між точками $\mathbf{x}$ та $\mathbf{y}$                   \\
    L1                          & Манхеттенська метрика відстані (L1-норма)                           \\
    L2                          & Евклідова метрика відстані (L2-норма)                               \\
    $u_{v,c}$                   & Корисність виборця $v$ для кандидата $c$                            \\
    $\mathbf{x}_v$              & Позиція виборця $v$ у просторі                                      \\
    $\mathbf{y}_c$              & Позиція кандидата $c$ у просторі                                    \\
    ЕЗВ                         & Ефективність задоволеності виборців (Voter Satisfaction Efficiency) \\
    РГ                         & Рангове голосування (Instant Runoff Voting, IRV)                     \\
   
\end{tabular}

\newpage

% --- ВСТУП ---
\section*{ВСТУП}
\addcontentsline{toc}{section}{ВСТУП}

Математичний аналіз виборчих процедур активно розвивається від XVIII століття: Жан-Шарль де Борда (1781) та маркіз де Кондорсе (1785) сформулювали перші ключові ідеї та парадокси, зокрема відомий парадокс циклічного голосування [4]. У середині XX століття Кеннет Ерроу у праці «Social Choice and Individual Values» довів теорему про неможливість і тим самим окреслив принципові межі демократичних процедур [1]. Подальший розвиток теорії привів до просторових моделей голосування: Дункан Блек [3] та Ентоні Даунс [5] запропонували описувати політичний спектр як метричний простір. Сучасні роботи, зокрема Дональда Саарі [12], використовують геометричні та топологічні підходи для пояснення відповідних парадоксів, проте питання того, як різні виборці можуть «вимірювати» політичну відстань по-різному, залишається дослідженим недостатньо повно.

Просторові моделі голосування є зручним інструментом для формалізації демократичних процесів, оскільки безпосередньо пов’язують індивідуальні переваги з геометрією ідеологічного простору. Актуальність цього дослідження визначається двома чинниками: по-перше, у високих вимірах проявляються ефекти концентрації міри, що змінюють «геометричну інтуїцію» та можуть впливати на результати; по-друге, стандартне припущення про єдину метрику відстані для всіх виборців є надто сильним і не завжди відповідає реальним відмінностям у сприйнятті політичної близькості.

\textbf{Об'єкт дослідження} --- просторові моделі голосування у багатовимірних ідеологічних просторах із гетерогенними метриками відстані. \textbf{Предмет дослідження} --- математичні механізми, через які гетерогенність метрик впливає на результати голосування, а також взаємодія між вимірністю простору, властивостями метрик і правилами агрегації переваг.

\textbf{Мета дослідження} --- провести комплексний математичний і обчислювальний аналіз гетерогенності метрик відстані у багатовимірних ідеологічних просторах та пояснити, як саме вона впливає на результати голосування за різних правил агрегації. Для досягнення мети поставлено такі завдання: обґрунтувати теоретичні передумови аналізу концентрації міри у високих вимірах і встановити зв’язок між вимірністю та центральністю виборців; побудувати обчислювальну модель для симуляції електоральних профілів із гетерогенними метриками у просторах різної вимірності; емпірично оцінити вплив гетерогенності метрик на результати для різних правил (відносна більшість, метод Борда, РГ, Кондорсе) та вимірів простору (1--10); запропонувати декомпозицію розбіжностей на сильні та периферійно-вирівняні компоненти; дослідити напрямкові асиметрії ефектів залежно від правила голосування і напрямку призначення метрик; оцінити ефективність задоволеності виборців (ЕЗВ) у гетерогенних метричних конфігураціях.

Дослідження спирається на поєднання теоретичного аналізу (зокрема, ідей концентрації міри та закону великих чисел) і обчислювальних експериментів (симуляції Монте-Карло, геометричний та декомпозиційний аналіз).

У просторових моделях голосування виборців і кандидатів задають як точки у багатовимірному ідеологічному просторі. Природне припущення полягає в тому, що виборець віддає перевагу тим кандидатам, які ближчі до його «ідеальної точки», а інтенсивність цієї переваги визначається метрикою відстані. Водночас у реальних умовах різні групи виборців можуть оцінювати політичну «дистанцію» не однаково: хтось зважує питання приблизно рівномірно (L2/Евклідова метрика), хтось сильніше реагує на сумарні відхилення або на окремі «болючі» координати (L1/Манхеттенська або Чебишева), а хтось більше орієнтується на ідеологічний напрям (Косинусна). На тлі ефектів високої вимірності це ставить практичне й теоретичне питання: \textit{Як гетерогенність метрик відстані впливає на результати голосування у високовимірних ідеологічних просторах, та які математичні механізми лежать в основі цих ефектів?}

Основний внесок роботи полягає у порівняльному аналізі метрик Мінковського (L1, L2, Чебишева) та напрямкової метрики (Косинусна) з урахуванням феномену концентрації міри. Показано, що після нормалізації центральність виборців у високовимірному гіперкубі $[-1,1]^d$ концентрується біля $\sqrt{1/3} \approx 0.577$. За результатами симуляцій встановлено, що ефекти гетерогенності зберігаються для всіх протестованих вимірів (1--10), а рівні розбіжностей лежать у межах 4--18.5\%. Також виявлено суттєві напрямкові асиметрії залежно від правила голосування (для методу Борда асиметрія сягає 51.5 п.п.). Запропоновано декомпозицію розбіжності на сильні та периферійно-вирівняні компоненти, що дозволяє точніше інтерпретувати механізм впливу гетерогенності в різних конфігураціях.

\newpage

% --- РОЗДІЛ 1 ---
\section{РОЗДІЛ 1}

У цьому розділі узагальнено теоретичні засади просторових моделей голосування та розглянуто ключові метрики відстані, які визначають індивідуальні переваги.

\subsection{Просторові моделі та метрики відстані}

Просторова теорія голосування, розвинена Енелоу та Хінічем [6], виходить із припущення, що кожен виборець має «ідеальну точку» в $d$-вимірному просторі політик, а корисність кандидата для цього виборця зменшується зі зростанням відстані між відповідними точками.

\subsubsection{Метрики Мінковського}

Ключовим елементом нашої постановки є сімейство метрик Мінковського ($L_p$-метрики), які задають відстань між точками $x = (x_1, \dots, x_d)$ та $y = (y_1, \dots, y_d)$ формулою:
\begin{align}
    d_p(\mathbf{x}, \mathbf{y}) & = \left(\sum_{i=1}^d |x_i - y_i|^p\right)^{1/p}, \quad p \geq 1
\end{align}

Наведемо коротку інтерпретацію основних випадків у контексті прийняття рішень:

\begin{itemize}
    \item \textbf{Манхеттенська метрика ($L_1$, $p=1$):} Визначається як сума модулів різниць координат. У виборчій інтерпретації це відповідає «логіці незалежних питань»: виборець оцінює кандидата по кожній осі окремо (наприклад, економіка, екологія) і підсумовує «штрафи» за невідповідність. Для $L_1$ характерно, що «компенсація» відхилення по одній осі зближенням по іншій виражена слабше, ніж у евклідовому випадку. Як зазначають Балінскі та Ларакі [2], подібні лінійні схеми природно виникають в задачах оцінювання незалежних критеріїв.
    
    \item \textbf{Евклідова метрика ($L_2$, $p=2$):} Класична відстань «по прямій». Вона допускає плавну взаємну компенсацію відхилень по різних осях: суттєвіша близькість в одних питаннях може частково нівелювати розбіжність в інших. Це найпоширеніший вибір у літературі, зокрема в роботах Леду [9], хоча він не завжди відбиває реалістичні способи оцінювання.
    
    \item \textbf{Метрика Чебишова ($L_\infty$, $p \to \infty$):} Визначається як максимум з різниць координат: $d_\infty(x, y) = \max_i |x_i - y_i|$. Її зручно інтерпретувати як модель «домінуючого критерію»: вирішальною стає найбільша невідповідність, і навіть добра близькість за іншими параметрами не компенсує одну критичну розбіжність.
\end{itemize}

Гетерогенність цих метрик у межах одного електорату може помітно змінювати колективний результат, зокрема положення «компромісних» альтернатив. Наприклад, Саарі у праці «Basic Geometry of Voting» [11] показує, як геометрія простору впливає на транзитивність переваг, однак здебільшого розглядає евклідів випадок. У цій роботі ми розширюємо постановку, допускаючи, що різні частини електорату використовують різні метрики (зокрема, поєднання $L_1$ та $L_2$).

\subsubsection{Напрямкові метрики: Косинусна відстань}

Косинусна відстань вимірює насамперед кутове розходження напрямків, а не «лінійну» просторову відстань:
\begin{align}
    d_{\text{cos}}(\mathbf{x}, \mathbf{y}) & = 1 - \frac{\mathbf{x} \cdot \mathbf{y}}{||\mathbf{x}|| \cdot ||\mathbf{y}||} = 1 - \cos(\theta)
\end{align}

де $\theta$ --- кут між векторами. Важлива властивість цієї метрики --- інваріантність до масштабування: ключовим є напрям, а не модуль. Тому вона природно моделює ідеологічне вирівнювання: точки з подібними напрямками (навіть якщо вони на різних відстанях від початку координат) мають малу косинусну відстань.

\subsection{Прокляття вимірності}

Високовимірні простори мають низку контрінтуїтивних геометричних властивостей. Для рівномірної вибірки в гіперкубі $[-1,1]^d$ справджуються такі типові твердження:

\begin{proposition}[Концентрація об'єму]
    Коли $d \to \infty$, майже весь об'єм $d$-вимірного гіперкуба концентрується біля його поверхні. Частка об'єму на відстані $\epsilon$ від межі наближається до 1.
\end{proposition}

\begin{proposition}[Рівномірність відстані]
    Для двох випадкових точок \\
    $\mathbf{x}, \mathbf{y} \sim \text{Uniform}([-1,1]^d)$, розподіл $||\mathbf{x} - \mathbf{y}||_2$ стає все більш концентрованим навколо середнього зі збільшенням $d$.
\end{proposition}

\subsubsection{Аналітичне наближення середньої відстані}

Хоча для малих вимірів (приблизно до 4--5) відомі точні формули для середньої відстані, зі зростанням $d$ їхній вивід швидко ускладнюється, тому для високих вимірів це практично непридатно. Далі ми користуємося асимптотичним наближенням. Для нормалізованих евклідових відстаней (поділених на $\sqrt{d}$) середнє прямує до $\sqrt{2/3}$ при $d \to \infty$, а для скінченних $d$ зручною є уточнена апроксимація:

\begin{align}
    \mathbb{E}\left[\frac{||\mathbf{x} - \mathbf{y}||_2}{\sqrt{d}}\right] \approx \sqrt{\frac{2}{3}} \left(1 - \frac{7}{40d}\right)
\end{align}

Ця формула враховує скінченновимірні поправки до асимптотичної межі. Із ростом $d$ доданок $7/(40d)$ зникає, і середнє наближається до $\sqrt{2/3} \approx 0.816497$. Похибка апроксимації найбільша у вимірі 1 (порядку \(\sim 1\%\), де точне значення дорівнює $2/3 = 0.66666\ldots$) і швидко зменшується зі зростанням вимірності.

\begin{figure}[h]
    \centering
    \includegraphics[width=\textwidth]{distance_uniformity_visualization.png}
    \caption{Емпіричне підтвердження Твердження 2.2: Розподіл нормалізованих Евклідових відстаней між випадковими парами точок у $[-1,1]^d$ для вимірів 1, 2, 3, 5, 10, 20, 50 та 100. Кожна гістограма показує щільність нормалізованих відстаней (поділених на $\sqrt{d}$), з червоними пунктирними лініями, що вказують емпіричне середнє, синіми пунктирними лініями, що показують аналітичне середнє $\sqrt{2/3}(1-7/(40d))$, та помаранчевими пунктирними лініями, що позначають $\pm 1$ стандартне відхилення. Зі збільшенням виміру розподіл стає все більш концентрованим навколо середнього, зі стандартним відхиленням, що зменшується від 0.471 (вимір 1) до 0.049 (вимір 100).}
    \label{fig:distance_uniformity}
\end{figure}

На рис.~\ref{fig:distance_uniformity} наведено емпіричну ілюстрацію відповідного ефекту: зі збільшенням вимірності розподіл нормалізованих відстаней (поділених на $\sqrt{d}$) концентрується навколо аналітичного середнього. У низьких вимірах (зокрема, при $d=1$) розподіл ще широкий і асиметричний, тоді як для $d=100$ він стає різко піковим; стандартне відхилення зменшується від 0.471 (вимір 1) до 0.049 (вимір 100).

\begin{figure}[h]
    \centering
    \includegraphics[width=\textwidth]{distance_variance_convergence.png}
    \caption{Збіжність дисперсії розподілу відстані з виміром. \textbf{Зліва}: Коефіцієнт варіації ($\sigma/\mu$) зменшується з виміром, показуючи, що відносна дисперсія зменшується як $1/\sqrt{d}$. \textbf{Справа}: Абсолютне стандартне відхилення (нормалізоване на $\sqrt{d}$) також зменшується з виміром. Обидва графіки використовують логарифмічну вісь x для показу збіжності через порядки величини.}
    \label{fig:variance_convergence}
\end{figure}

На рис.~\ref{fig:variance_convergence} цю збіжність показано кількісно: побудовано коефіцієнт варіації (стандартне відхилення, поділене на середнє) та абсолютне стандартне відхилення як функції $d$. Обидві величини зменшуються зі зростанням вимірності, що узгоджується з поведінкою $1/\sqrt{d}$ для відносної дисперсії та загалом відображає феномен концентрації міри.

Узагальнено кажучи, ці явища створюють «порожні кути» --- області простору, які геометрично можливі, але статистично малоймовірні для розміщення виборців або кандидатів. Для голосування це важливо тим, що зі зростанням $d$ «ефективний» простір конфігурацій звужується, а багато статистичних характеристик стають більш передбачуваними.

\subsection{Концентрація міри та ефективний радіус}

Ключовий теоретичний результат цього підрозділу стосується концентрації центральності виборців у високих вимірах.

\begin{definition}[Нормалізована L2 центральність]
    Для виборця з позицією $\mathbf{x} \in [-1,1]^d$ та центром $\mathbf{c}=\mathbf{0}$ (геометричний центр) нормалізована L2-центральність визначається як
    \begin{align}
        c(\mathbf{x}) = \frac{||\mathbf{x}||_2}{\sqrt{d} \cdot \frac{\text{range}}{2}} = \frac{||\mathbf{x}||_2}{\sqrt{d}}
    \end{align}
    де знаменник відповідає половині діагоналі гіперкуба (для інтервалу $[-1,1]$).
\end{definition}

\begin{theorem}[Концентрація центральності]
    Для $\mathbf{X} \sim \text{Uniform}([-1,1]^d)$, коли $d \to \infty$:
    \begin{align}
        \frac{||\mathbf{X}||_2}{\sqrt{d}} \xrightarrow{p} \sqrt{\mathbb{E}[X_1^2]} = \sqrt{\frac{1}{3}} \approx 0.577
    \end{align}
    де $\mathbb{E}[X_1^2]$ --- математичне сподівання квадрата координати.
\end{theorem}

\begin{proof}
    За законом великих чисел:
    \begin{align}
        \frac{||\mathbf{X}||_2^2}{d} = \frac{1}{d}\sum_{i=1}^d X_i^2 \xrightarrow{p} \mathbb{E}[X_1^2] = \int_{-1}^1 \frac{x^2}{2} dx = \frac{1}{3}
    \end{align}
    Взявши квадратні корені та застосувавши неперервність функції квадратного кореня, отримуємо результат.
\end{proof}

Теорема пояснює, чому класифікація «центр/периферія» зручно формулюється через перцентилі: у високих вимірах майже всі значення центральності скупчуються поблизу $0.577$, тому фіксовані радіальні пороги втрачають універсальність, тоді як перцентильні пороги адаптуються до розподілу.

\subsubsection{Емпіричне підтвердження}

Симуляції узгоджуються з цим теоретичним висновком. У табл.~\ref{tab:centrality_convergence} наведено середню центральність за вимірами для рівномірної вибірки в гіперкубі:

\begin{table}[h]
    \centering
    \caption{Емпірична збіжність центральності до $\sqrt{1/3} \approx 0.577$}
    \label{tab:centrality_convergence}
    \begin{tabular}{lcc}
        \toprule
        Вимір & Середня центральність & Відхилення від теорії \\
        \midrule
        1     & 0.4995                & 0.0778                \\
        2     & 0.5410                & 0.0363                \\
        3     & 0.5541                & 0.0232                \\
        5     & 0.5645                & 0.0128                \\
        10    & 0.5712                & 0.0062                \\
        20    & 0.5745                & 0.0029                \\
        50    & 0.5762                & 0.0012                \\
        100   & 0.5768                & 0.0006                \\
        200   & 0.5770                & 0.0003                \\
        \bottomrule
    \end{tabular}
\end{table}

Збіжність є наочною: вже при $d=200$ середня центральність (0.5770) відрізняється від теоретичного значення (0.5774) лише на 0.0003. Для $d \geq 10$ середня центральність дорівнює 0.5751 із відхиленням 0.0022. Отже, у високих вимірах позиції виборців концентруються біля «ефективного радіуса», і практичне розрізнення між більш центральними та більш периферійними виборцями природно будувати на основі перцентилів.

\subsection{Висновки до розділу 1}

У першому розділі викладено теоретичні засади просторових моделей голосування та проаналізовано властивості основних метрик відстані у багатовимірних середовищах. Підкреслено відмінність між метриками Мінковського ($L_1, L_2, L_\infty$), які описують «лінійну» просторову близькість, та косинусною мірою, що фіксує передусім ідеологічний напрям.

На основі ідей концентрації міри обґрунтовано, що зі зростанням вимірності розподіл відстаней дедалі сильніше концентрується, а центральність виборців збігається до $\sqrt{1/3} \approx 0.577$. Це мотивує використання адаптивної класифікації виборців (зокрема, за перцентилями) і створює основу для коректного моделювання гетерогенних електоральних профілів у наступних розділах.

\newpage

% --- РОЗДІЛ 2 ---
\section{РОЗДІЛ 2}

У цьому розділі описано методологію симуляцій, нормалізацію корисності та інструменти аналізу впливу метричної гетерогенності.

\subsection{Рамки Монте-Карло для електоральних симуляцій}

Симуляційна частина роботи реалізує підхід Монте-Карло до електорального моделювання та задає єдині параметри експериментів:

\begin{itemize}
    \item \textbf{Генерація профілів}: для кожної експериментальної конфігурації генерується $N=200$ незалежних електоральних профілів.
    \item \textbf{Просторова вибірка}: позиції виборців і кандидатів вибираються рівномірно з $[-1,1]^d$, що забезпечує повний кутовий діапазон для косинусної відстані.
    \item \textbf{Кількість виборців}: у базових експериментах використовується $n=100$ виборців на профіль; додатково проводяться тести масштабування від 10 до 500 виборців.
    \item \textbf{Кількість кандидатів}: фіксується на рівні $M=5$ кандидатів на профіль.
    \item \textbf{Випадкові насіння}: для забезпечення статистичної незалежності кожен запуск використовує окреме насіння (42, 43, 44, \ldots); для одноразових візуалізацій застосовується фіксоване насіння 42 з метою відтворюваності.
\end{itemize}

Для кожного профілю обчислюються такі величини:
\begin{enumerate}
    \item Позиції виборців: $\mathbf{X} \in \mathbb{R}^{n \times d}$
    \item Позиції кандидатів: $\mathbf{Y} \in \mathbb{R}^{M \times d}$
    \item Матриця відстаней: $D \in \mathbb{R}^{n \times M}$ (за гетерогенних або однорідних метрик)
    \item Матриця корисності: $U \in \mathbb{R}^{n \times M}$ (через лінійну нормалізацію)
    \item Ранжування: $R \in \{0,\ldots,M-1\}^{n \times M}$ (через впорядкування корисностей, тобто argsort)
    \item Результати правил голосування: переможці для методів відносної більшості, Борда, РГ та Кондорсе
\end{enumerate}

\subsection{Відносна нормалізація корисності: Масштабування найкраще-найгірше}

Порівнювати корисності між різними виборцями в просторових моделях складно, адже за гетерогенних метрик одна й та сама «геометрична» різниця може сприйматися по-різному. Щоб зробити агрегування коректним, ми застосовуємо нормалізацію корисності окремо для кожного виборця, відображаючи значення в інтервал $[0,1]$ за допомогою \textit{лінійного} перетворення. У межах цієї роботи свідомо використовується лінійна форма корисності, без залучення гаусових чи інших нелінійних функцій, аби не змішувати вплив метрики з впливом форми корисності.

\begin{definition}[Лінійна корисність з масштабуванням найкраще-найгірше]
    Для виборця $v$ в позиції $\mathbf{x}_v$ та кандидата $c$ в позиції $\mathbf{y}_c$, ми спочатку обчислюємо сирі корисності з відстаней:
    \begin{align}
        \tilde{u}_{v,c} = -d(\mathbf{x}_v, \mathbf{y}_c)
    \end{align}
    де $d$ позначає метрику відстані, призначену виборцю $v$ (яка може варіюватися між виборцями в гетерогенних метричних конфігураціях). Ці сирі корисності потім нормалізуються на виборця через масштабування найкраще-найгірше:
    \begin{align}
        u_{v,c} = \frac{\tilde{u}_{v,c} - \min_{c'} \tilde{u}_{v,c'}}{\max_{c'} \tilde{u}_{v,c'} - \min_{c'} \tilde{u}_{v,c'}}
    \end{align}
    У результаті цього перетворення найбільш переважений кандидатом виборця (тобто той, що мінімізує відстань у його метриці) отримує корисність, рівну одиниці, тоді як найменш переважений кандидат має корисність нуль. Значення для решти альтернатив інтерполюються лінійно відповідно до їх відносних відстаней.
\end{definition}

Запропонована схема гарантує обмеженість корисностей та дозволяє агрегувати їх між виборцями, не «стираючи» різницю у метричних припущеннях на індивідуальному рівні. Лінійний характер перетворення допомагає уникнути додаткових нелінійних ефектів (типових, наприклад, для гаусових моделей) і тим самим краще ізолювати вплив саме метричної гетерогенності. Оскільки нормалізація виконується відносно множини кандидатів у кожному конкретному профілі, отримані корисності не залежать від абсолютного масштабу простору, що особливо важливо у багатовимірних середовищах.

\subsubsection{Декомпозиція міри розбіжності}

Щоб краще інтерпретувати вплив гетерогенності метрик на підсумок виборів, ми розкладаємо загальну розбіжність на два змістовні компоненти:

\begin{definition}[Сильна розбіжність]
    Сильна розбіжність $D_{\text{strong}}$ --- це відсоток профілів, де гетерогенний переможець відрізняється від \textit{обидвох} базових переможців метрики центру \textit{та} метрики периферії:
    \begin{align}
        D_{\text{strong}} = \frac{1}{N} \sum_{p=1}^N \mathbf{1}[w_{\text{het},p} \neq w_{\text{center},p} \land w_{\text{het},p} \neq w_{\text{periphery},p}]
    \end{align}
    де $w_{\text{het},p}$, $w_{\text{center},p}$ та $w_{\text{periphery},p}$ --- переможці для профілю $p$ за гетерогенних, однорідних метрики центру та однорідних метрики периферії умов відповідно.
\end{definition}

Сильна розбіжність відповідає ситуаціям, коли змішані метрики фактично \textit{породжують} новий результат, який не відтворюється жодною з однорідних базових ліній.

\begin{definition}[Периферійно-вирівняна розбіжність]
    Периферійно-вирівняна розбіжність $D_{\text{per-align}}$ --- це відсоток профілів, де гетерогенний переможець дорівнює базовому переможцю метрики периферії, але відрізняється від базового метрики центру:
    \begin{align}
        D_{\text{per-align}} = \frac{1}{N} \sum_{p=1}^N \mathbf{1}[w_{\text{het},p} = w_{\text{periphery},p} \land w_{\text{het},p} \neq w_{\text{center},p}]
    \end{align}
\end{definition}

Периферійно-вирівняна розбіжність фіксує випадки, коли гетерогенність \textit{зсуває} результат у бік метрики периферії: гетерогенний переможець збігається з «периферійною» базовою лінією, хоча «центральна» базова лінія обрала б іншого кандидата.

Таким чином, загальна розбіжність відносно базової лінії центру має вигляд $D_{\text{total}} = D_{\text{strong}} + D_{\text{per-align}}$.

\subsection{Математична логіка правил агрегації}

\subsubsection{Метод Борда}

Метод Борда нараховує бали залежно від позиції кандидата у ранжуванні виборця:
\begin{align}
    \text{score}_c = \sum_{v=1}^n (M - \text{rank}_v(c))
\end{align}
де $\text{rank}_v(c) \in \{0,\ldots,M-1\}$ --- ранг кандидата $c$ у порядку переваг виборця $v$ (0 = найбільш переважений).

Переможець: $\arg\max_c \text{score}_c$

\subsubsection{Ранжовані пари (метод Кондорсе)}

Метод ранжованих пар будує соціальне впорядкування так:
\begin{enumerate}
    \item Обчислення попарних марж: $m_{i,j} = |\{v: i \succ_v j\}| - |\{v: j \succ_v i\}|$
    \item Сортування пар $(i,j)$ за спаданням маржі $m_{i,j}$
    \item Фіксація пар по порядку, пропускаючи пари, які створили б цикли
    \item Переможець --- найкращий кандидат у фінальному впорядкуванні
\end{enumerate}

Метод задовольняє критерій Кондорсе: якщо існує переможець Кондорсе, ранжовані пари вибирають саме його.

\subsubsection{Голосування за оцінкою (кардинальне)}

Голосування за оцінкою використовує нормалізовані корисності безпосередньо:
\begin{align}
    \text{score}_c = \sum_{v=1}^n u_{v,c}
\end{align}
Переможець: $\arg\max_c \text{score}_c$

За нашої нормалізації це еквівалентно максимізації утилітарного соціального добробуту.

\subsubsection{Відносна більшість та РГ}

\begin{itemize}
    \item \textbf{Відносна більшість}: Переможець --- кандидат, ранжований першим найбільшою кількістю виборців.
    \item \textbf{Рангове голосування (РГ)}: ітеративно усуває кандидата з найменшою кількістю голосів першого місця, перерозподіляючи голоси доти, доки один кандидат не набере більшості.
\end{itemize}

\subsection{Збіжність вимірності}

Емпіричні результати показують, що ефекти метричної гетерогенності зберігаються для всіх протестованих вимірів, а рівні розбіжностей не демонструють простої монотонної тенденції зі зростанням $d$. Декомпозиція на сильні та периферійно-вирівняні компоненти дозволяє побачити, що механізм розбіжності змінюється залежно від виміру та правила голосування.

\subsubsection{Емпіричні докази}

У табл.~\ref{tab:dimensional_convergence} наведено декомпозовані рівні розбіжностей для пари метрик L2--Косинусна у вимірах 1--10:

\begin{table}[h]
    \centering
    \caption{Рівні розбіжностей за виміром (L2 центр, Косинусна периферія, поріг=0.5). Всі значення показують Сильна / Периферійно-вирівняна / Загальна розбіжність (\%).}
    \label{tab:dimensional_convergence}
    \begin{tabular}{lccc}
        \toprule
        Вимір & Відносна більшість & Борда            & РГ               \\
        \midrule
        1     & 6.5 / 12.0 / 18.5  & 6.5 / 0.5 / 7.0  & 10.5 / 5.5 / 16.0 \\
        2     & 6.5 / 7.5 / 14.0   & 5.0 / 3.5 / 8.5  & 6.0 / 6.5 / 12.5  \\
        3     & 2.5 / 7.0 / 9.5    & 1.5 / 2.5 / 4.0  & 1.0 / 9.0 / 10.0  \\
        4     & 1.0 / 7.5 / 8.5    & 8.0 / 3.5 / 11.5 & 3.5 / 10.0 / 13.5 \\
        5     & 4.0 / 7.0 / 11.0   & 2.0 / 6.5 / 8.5  & 4.5 / 13.0 / 17.5 \\
        7     & 5.0 / 13.0 / 18.0  & 6.0 / 8.0 / 14.0 & 1.5 / 10.0 / 11.5 \\
        10    & 3.0 / 9.5 / 12.5   & 4.5 / 6.0 / 10.5 & 4.0 / 9.0 / 13.0  \\
        \bottomrule
    \end{tabular}
\end{table}

Дані демонструють помітний вплив гетерогенності метрик у всіх вимірах; ключові спостереження такі:

\begin{enumerate}
    \item \textbf{Загальна розбіжність}: коливається в межах 4--18.5\% залежно від виміру та правила. Борда дає нижчі значення (4--14\%), тоді як відносна більшість і РГ частіше демонструють вищі рівні (8.5--18.5\%).

    \item \textbf{Патерни декомпозиції}: співвідношення сильних і периферійно-вирівняних розбіжностей істотно змінюється:
          \begin{itemize}
              \item Борда, вимір 1: 92.9\% сильних розбіжностей (гетерогенність частіше дає нові результати)
              \item РГ, вимір 3: 10.0\% сильних розбіжностей (розбіжність переважно є периферійно-вирівняною)
          \end{itemize}

    \item \textbf{Відсутність монотонності}: розбіжність не спадає монотонно з ростом $d$, що свідчить про взаємодію між геометрією, метрикою та механізмом агрегації, а не про «просту» збіжність.
\end{enumerate}

У табл.~\ref{tab:metric_pairs} наведено загальні рівні розбіжностей для всіх пар метрик у вимірі 2:

\begin{table}[h]
    \centering
    \caption{Загальні рівні розбіжностей пар метрик у вимірі 2 (нотація A $\rightarrow$ B).}
    \label{tab:metric_pairs}
    \begin{tabular}{lccc}
        \toprule
        Пара метрик                      & Відносна більшість & Борда           & РГ    \\
        \midrule
        L2 $\rightarrow$ Косинусна       & 14.0\%             & 8.5\%           & 12.5\% \\
        Косинусна $\rightarrow$ L2       & 16.0\%             & \textbf{60.0\%} & 4.5\%  \\
        L1 $\rightarrow$ Косинусна       & 16.5\%             & 8.5\%           & 11.5\% \\
        Косинусна $\rightarrow$ L1       & 14.0\%             & \textbf{52.5\%} & 9.5\%  \\
        Косинусна $\rightarrow$ Чебишева & 20.0\%             & \textbf{64.5\%} & 8.0\%  \\
        Чебишева $\rightarrow$ Косинусна & 16.0\%             & 11.5\%          & 12.5\% \\
        L1 $\rightarrow$ L2              & 14.0\%             & 11.5\%          & 9.0\%  \\
        L2 $\rightarrow$ L1              & 9.5\%              & 5.5\%           & 6.5\%  \\
        L1 $\rightarrow$ Чебишева        & 19.0\%             & 18.0\%          & 11.0\% \\
        Чебишева $\rightarrow$ L1        & 13.0\%             & 11.0\%          & 10.5\% \\
        L2 $\rightarrow$ Чебишева        & 7.0\%              & 7.5\%           & 6.5\%  \\
        Чебишева $\rightarrow$ L2        & 7.0\%              & 7.0\%           & 7.5\%  \\
        \bottomrule
    \end{tabular}
\end{table}

Таблиця виявляє \textbf{виражену асиметрію}: для пар типу «Косинусна--Мінковський» розбіжності суттєво відрізняються від пар у зворотному напрямку, причому ефект \textit{залежить від правила} голосування. Найбільш помітно це проявляється для методу Борда, де різниця між напрямками перевищує 50 процентних пунктів:

\begin{itemize}
    \item \textbf{Борда}: демонструє сильну чутливість до того, яка метрика призначається центральним або периферійним виборцям. Для Косинусна$\rightarrow$L2 розбіжність становить 60.0\% проти 8.5\% для L2$\rightarrow$Косинусна (асиметрія 51.5 п.п.). Аналогічно, Косинусна$\rightarrow$Чебишева дає 64.5\% проти 11.5\% для Чебишева$\rightarrow$Косинусна (асиметрія 53.0 п.п.). Це узгоджується з тим, що косинусна відстань істотно змінює агрегування Борда, коли застосовується до більшості (центральної групи).

    \item \textbf{РГ}: показує помірну асиметрію зі зворотним напрямком. L2$\rightarrow$Косинусна має 12.5\% розбіжності проти 4.5\% для Косинусна$\rightarrow$L2 (асиметрія 8.0 п.п.).

    \item \textbf{Відносна більшість}: загалом є більш симетричною (14--20\% розбіжності), а асиметрії зазвичай не перевищують 6 п.п.; водночас абсолютні рівні розбіжності часто вищі, ніж у РГ (окрім пари Чебишева$\rightarrow$L2).
\end{itemize}

Особливо показовою є пара Борда--Косинусна: коли Косинусна є метрикою центру, результати відрізняються від базової лінії L2 у 60\% профілів, тоді як для L2 як метрики центру розбіжність знижується до 8.5\%. Отже, у цій постановці саме \textit{призначення} метрик групам виборців виявляється не менш важливим, ніж вибір метрик як таких, а механізм підсумовування рангових переваг у Борда істотно взаємодіє з напрямковою природою косинусної відстані.

\subsubsection{Аналіз декомпозиції: Сильна проти периферійно-вирівняної розбіжності}

Декомпозиція розбіжності на сильні та периферійно-вирівняні компоненти дозволяє відокремити \textit{механізм} впливу гетерогенності:

\begin{table}[h]
    \centering
    \caption{Декомпозиція розбіжності для L2$\rightarrow$Косинусна (Вимір 2)}
    \label{tab:decomposition}
    \begin{tabular}{lccc}
        \toprule
        Правило            & Сильна & Периферійно-вирівняна & Сильна\% \\
        \midrule
        Відносна більшість & 6.5\%  & 7.5\%                  & 46.4\%   \\
        Борда              & 5.0\%  & 3.5\%                  & 58.8\%   \\
        РГ                & 6.0\%  & 6.5\%                  & 48.0\%   \\
        \bottomrule
    \end{tabular}
\end{table}

\textbf{Ключові результати:}
\begin{itemize}
    \item \textbf{Борда} має найбільшу частку сильних розбіжностей (58.8\%), тобто гетерогенність частіше призводить до результатів, які не пояснюються жодною з двох базових ліній.

    \item \textbf{Відносна більшість та РГ} мають більш збалансовану структуру (~46--48\% сильних), тобто розбіжність приблизно порівну складається з «нових» результатів і периферійних вирівнювань.

    \item Для Косинусна$\rightarrow$L2 (Борда) сильна розбіжність дорівнює 11.5\% (19.2\% від загальної), тоді як периферійно-вирівняна --- 48.5\% (80.8\% від загальної). Отже, висока загальна розбіжність (60\%) тут переважно зумовлена вирівнюванням із косинусно-однорідною базовою лінією, а не появою принципово нових переможців.

    \item Навпаки, для L2$\rightarrow$Косинусна (Борда) сильна компонента становить 5.0\% із 8.5\% загальної (тобто 58.8\%), що вказує на більшу частку справді «емерджентних» результатів.
\end{itemize}

Отже, гетерогенність є критично залежною і від правила голосування, і від напрямку призначення метрик. Та сама пара (L2--Косинусна) може діяти по-різному: L2 як метрика центру частіше веде до появи нових переможців, тоді як Косинусна як метрика центру переважно підсилює косинусно-однорідну базову лінію.

\subsubsection{Візуалізація сильних розбіжностей}

Щоб проілюструвати механізм сильної розбіжності, на рис.~\ref{fig:strong_disagreement} показано один і той самий електоральний профіль у трьох метричних конфігураціях. Візуалізація демонструє, що незмінна просторова розстановка виборців і кандидатів може давати різних переможців залежно від того, як саме призначено метрики.

\begin{figure}[h]
    \centering
    \includegraphics[width=\textwidth]{strong_disagreement_visualization.png}
    \caption{Приклад сильної розбіжності для одного профілю в трьох метричних конфігураціях. \textbf{Зліва}: гетерогенна (L2 центр, Косинусна периферія) обирає Кандидата 3. \textbf{По центру}: однорідна L2 обирає Кандидата 2. \textbf{Справа}: однорідна Косинусна обирає Кандидата 0. Сині кола відповідають центральним виборцям (L2), червоні трикутники --- периферійним (Косинусна), пунктирне коло позначає поріг «центр/периферія». Кандидат 3 перемагає лише за гетерогенності, що ілюструє появу результату, не відтворюваного жодною однорідною базовою лінією.}
    \label{fig:strong_disagreement}
\end{figure}

Головний висновок із цієї візуалізації полягає в тому, що Кандидат 3 перемагає \textit{лише} за змішаних метрик; жодна однорідна базова лінія такого результату не дає:
\begin{itemize}
    \item Якби всі виборці використовували L2 (евклідова відстань), переміг би Кандидат 2
    \item Якби всі виборці використовували Косинусну (кутова відстань), переміг би Кандидат 0
    \item За змішаних метрик (L2 для центру, Косинусна для периферії) перемагає Кандидат 3
\end{itemize}

Це і є «просторовий підпис» сильної розбіжності: гетерогенний переможець не пояснюється жодною однорідною базовою лінією. Взаємодія між центральними виборцями (евклідове сприйняття) та периферійними (напрямкове сприйняття) приводить до нового колективного результату, який зникає, якщо всіх привести до однієї метрики.

\subsubsection{Теоретичне пояснення}

Те, що ефекти гетерогенності зберігаються в різних вимірах попри концентрацію міри, означає: вибір метрики залишається значущим навіть у високовимірних просторах. Концентрація призводить до того, що більшість виборців розташовуються поблизу ефективного радіуса ($\approx 0.577$), однак \textit{напрямкові} властивості косинусної відстані змінюють відносні ранги альтернатив так, що ці відмінності не зникають від простого масштабування. Немонотонні зміни рівнів розбіжностей (наприклад, для Борда: 11.5\% у вимірі 4 проти 4.0\% у вимірі 3) вказують на взаємодію між вимірністю, властивостями метрик та механізмами агрегації, яку не можна звести до простого аргументу «збіжності з ростом $d$».

\subsection{Висновки до розділу 2}

У другому розділі описано методологію симуляцій Монте-Карло та введено інструменти аналізу, зокрема нормалізацію корисності й декомпозицію розбіжності. Показано, що ефекти гетерогенності метрик зберігаються для всіх протестованих вимірів (1--10), а загальні рівні розбіжностей лежать у межах 4--18.5\%. Виявлено виражену взаємодію Борда з косинусною метрикою: асиметрія залежно від напрямку призначення метрик сягає 51.5 п.п., що підкреслює важливість того, \textit{кому} саме призначено ту чи іншу метрику. Декомпозиція на сильні та периферійно-вирівняні компоненти показує, що механізм розбіжності змінюється залежно від правила та конфігурації; найбільшу напрямкову чутливість демонструє Борда, тоді як відносна більшість є ближчою до симетричної поведінки.

\newpage

% --- РОЗДІЛ 3 ---
\section{РОЗДІЛ 3}

У цьому розділі узагальнено результати експериментів і проаналізовано, як метрична гетерогенність змінює підсумки голосування та характер розбіжностей.

\subsection{Вплив метрик на парадокси Кондорсе}

\subsubsection{Взаємодія Борда-Косинусна}

Симуляції виявляють важливий феномен: коли косинусна відстань задається як метрика для центральної групи виборців у поєднанні з методом Борда, рівні розбіжностей істотно зростають порівняно з конфігураціями, де метрикою центру є L2.

Для конфігурації Косинусна$\rightarrow$L2 (Борда) отримано такі характеристики:
\begin{itemize}
    \item Загальна розбіжність: 60.0\% (проти 8.5\% для L2$\rightarrow$Косинусна), асиметрія 51.5 п.п.
    \item Сильна розбіжність: 11.5\% (19.2\% від загальної)
    \item Периферійно-вирівняна розбіжність: 48.5\% (80.8\% від загальної)
    \item Отже, основний внесок у загальну розбіжність робить вирівнювання з косинусно-однорідною базовою лінією, а не поява принципово нових переможців
\end{itemize}

Ця «взаємодія Борда--Косинусна» виявляє принципову асиметрію: коли Косинусна призначена центральним виборцям, гетерогенний результат часто збігається з тим, що дало б косинусно-однорідне голосування (48.5\% профілів). Це свідчить, що напрямкова природа косинусної відстані (орієнтація на ідеологічний напрям замість «лінійної» близькості) істотно взаємодіє з механізмом підсумовування рангових переваг у Борда і може посилювати вплив косинусної метрики, коли вона застосовується до більшості виборців.

Зворотний напрямок (L2$\rightarrow$Косинусна) демонструє інший механізм: периферійно-вирівняна розбіжність становить лише 3.5\%, тоді як сильна --- 5.0\% (58.8\% від загальної). Тобто, коли метрикою центру є L2, розбіжність частіше пов’язана з появою нових переможців, а не з простим «зсувом» до базової лінії периферії.

\subsubsection{Взаємодії метрик Мінковського}

Для пар метрик L1 та L2 спостерігається помірна чутливість до гетерогенності з відносно симетричними напрямковими ефектами. Для пари L1--L2 у вимірі 2 маємо:
\begin{itemize}
    \item L1$\rightarrow$L2: 14.0\% (Відносна більшість), 11.5\% (Борда), 9.0\% (РГ)
    \item L2$\rightarrow$L1: 9.5\% (Відносна більшість), 5.5\% (Борда), 6.5\% (РГ)
    \item Асиметрія: 4.5 п.п. (Відносна більшість), 6.0 п.п. (Борда), 2.5 п.п. (РГ)
\end{itemize}

Декомпозиція показує, що для L1--L2 розбіжність часто є переважно периферійно-вирівняною (наприклад, L1$\rightarrow$L2 для Борда: 1.0\% сильних і 10.5\% периферійно-вирівняних). Тобто гетерогенність частіше зсуває результат у бік метрики периферії, а не створює нові «рівноваги». Це контрастує з косинусними взаємодіями, де напрямок призначення метрик має набагато більшу вагу для Борда.

Пари Чебишева--L2 демонструють майже ідеальну симетрію (7.0--7.5\% розбіжності в обох напрямках), що свідчить про відносно збалансовану взаємодію цих метрик і слабку залежність від напрямку призначення.

Зокрема, коли 50\% агентів використовують $L_1$, а 50\% --- $L_2$, «ядро» (множина точок, які не можуть бути переможені більшістю) часто стає порожнім навіть у двовимірному просторі. Це узгоджується з теоретичними передбаченнями МакКелві та Ордешкі, згаданими у [10], щодо хаотичності траєкторій голосування у багатовимірних моделях.

Окремо спостерігається зростання маніпулятивності систем голосування за гетерогенних метрик. Це узгоджується з висновками Гіббарда та Саттертуейта, узагальненими у [8], про неможливість повного захисту від стратегічного голосування, і додає уточнення: стратегія виборця залежить не лише від його позиції, а й від того, якою метрикою він вимірює «відстань» до альтернатив. Подібні постановки обговорюють Прокаччіа та інші [7] у контексті ШІ, де агенти можуть мати різні функції втрат.

\subsection{Перевірки ефективності задоволеності виборців (ЕЗВ)}

Ефективність задоволеності виборців оцінює, наскільки добре правило голосування наближає вибір до соціально оптимального кандидата:

\begin{definition}[ЕЗВ]
    Для корисностей $U \in \mathbb{R}^{n \times M}$ та переможця $w$:
    \begin{align}
        \text{ЕЗВ} = \frac{\bar{u}_w - \bar{u}_{\min}}{\bar{u}_{\max} - \bar{u}_{\min}}
    \end{align}
    де $\bar{u}_c = \frac{1}{n}\sum_{v=1}^n u_{v,c}$ --- середня корисність для кандидата $c$.
\end{definition}

ЕЗВ набуває значень від 0 (переможець --- найгірший кандидат) до 1 (переможець --- оптимальний кандидат).

\subsubsection{Порівняльна продуктивність за правилами}

У табл.~\ref{tab:ЕЗВ_by_dimension} наведено значення ЕЗВ для різних правил голосування за вимірами:

\begin{table}[h]
    \centering
    \caption{ЕЗВ за виміром (L2-Косинусна, гетерогенна).}
    \label{tab:ЕЗВ_by_dimension}
    \begin{tabular}{lccc}
        \toprule
        Вимір & Відносна більшість & Борда & РГ   \\
        \midrule
        1     & 0.606              & 0.966 & 0.581 \\
        2     & 0.821              & 0.989 & 0.584 \\
        3     & 0.834              & 0.994 & 0.528 \\
        4     & 0.848              & 0.990 & 0.536 \\
        5     & 0.883              & 0.992 & 0.530 \\
        7     & 0.887              & 0.995 & 0.581 \\
        10    & 0.909              & 0.993 & 0.633 \\
        \bottomrule
    \end{tabular}
\end{table}

Ключові спостереження:
\begin{enumerate}
    \item \textbf{Борда демонструє найвищі значення}: ЕЗВ $\geq 0.99$ для всіх вимірів, тобто стабільно високий рівень «соціального добробуту» незалежно від $d$.
    \item \textbf{Відносна більшість покращується з вимірністю}: ЕЗВ зростає від 0.606 (вимір 1) до 0.909 (вимір 10), що вказує на кращу здатність правил типу plurality ідентифікувати соціально кращого кандидата у вищих вимірах.
    \item \textbf{РГ є менш стабільним}: ЕЗВ змінюється від 0.528 (вимір 3) до 0.633 (вимір 10) і загалом є нижчим, ніж у Борда; найменше значення спостерігається у вимірі 3.
    \item \textbf{Загальна тенденція}: зі збільшенням $d$ ЕЗВ зростає або залишається стабільною для всіх розглянутих правил, хоча сила ефекту залежить від правила.
\end{enumerate}

\subsubsection{Ефекти гетерогенності метрик на ЕЗВ}

Для пари L2--Косинусна різниці ЕЗВ між гетерогенними та однорідними (базова лінія метрики центру) умовами є малими для всіх вимірів і правил, зазвичай у межах $\pm 0.01$. Це означає, що гетерогенність може помітно змінювати \textit{переможця} (що видно з рівнів розбіжностей), але має мінімальний вплив на \textit{якість} переможця, виміряну через соціальний добробут.

Водночас декомпозиція підкреслює важливий нюанс: коли Косинусна є метрикою центру, гетерогенний результат часто вирівнюється з косинусно-однорідною базовою лінією (48.5\% периферійно-вирівняної розбіжності для Косинусна$\rightarrow$L2 у Борда). Таке вирівнювання може частково пояснювати, чому різниці ЕЗВ лишаються малими: гетерогенне правило часто обирає переможців, близьких за якістю до тих, що оптимізують косинусно-базований соціальний добробут, який у багатьох профілях є подібним до L2-базованого.

\subsection{Висновки до розділу 3}

У третьому розділі узагальнено, як метрична гетерогенність змінює результати голосування. Показано виражені напрямкові асиметрії, передусім для методу Борда: взаємодія з косинусною метрикою дає 7-кратну різницю в рівнях розбіжності залежно від того, чи є Косинусна метрикою центру або периферії. Декомпозиція підтверджує, що механізм розбіжності критично залежить і від правила, і від напрямку призначення метрик. Водночас метрика майже не впливає на «якість» переможця в термінах ЕЗВ, попри те, що може помітно змінювати того, \textit{хто саме} перемагає.

\newpage

% --- ВИСНОВКИ ---
\section{ВИСНОВКИ}
\addcontentsline{toc}{section}{ВИСНОВКИ}

У цій роботі виконано комплексний математичний та обчислювальний аналіз впливу гетерогенності метрик відстані в багатовимірних ідеологічних просторах. Дослідження поєднує теоретичний розгляд концентрації міри у високих вимірах із симуляціями Монте-Карло (виміри 1--10) та систематичним порівнянням результатів для різних правил агрегації.

\subsection{Основні результати дослідження}

Результати дослідження є самостійними та отримані автором у межах теоретичного й обчислювального аналізу. Теоретичні твердження обґрунтовано математично, а емпіричні висновки отримано за допомогою симуляцій Монте-Карло з використанням власного програмного забезпечення.

Отримані результати показують, що ефекти метричної гетерогенності є \textit{залежними від правила} та \textit{напрямку призначення метрик}, і зберігаються навіть у високих вимірах:

\begin{enumerate}
    \item \textbf{Стійкість ефектів}: рівні розбіжностей 4--18.5\% зберігаються для всіх протестованих вимірів (1--10), що суперечить гіпотезі повного «зникнення» ефектів гетерогенності зі зростанням $d$.

    \item \textbf{Взаємодія Борда--Косинусна}: коли Косинусна є метрикою центру у методі Борда, розбіжність досягає 60\% (проти 8.5\% для L2 як метрики центру), тобто асиметрія становить 51.5 п.п. Це найбільший зафіксований напрямковий ефект.

    \item \textbf{Декомпозиція пояснює механізм}: частка сильної розбіжності (коли гетерогенний переможець відрізняється від обох базових ліній) варіюється від 10 до 93\% загальної розбіжності залежно від правила та конфігурації. Для Борда частка сильної компоненти суттєво різниться за напрямком: L2$\rightarrow$Косинусна має 58.8\% сильних розбіжностей, тоді як Косинусна$\rightarrow$L2 --- лише 19.2\%, тобто переважає периферійне вирівнювання. Цей напрямковий зсув у механізмі не менш важливий, ніж різниця у самій величині розбіжності.

    \item \textbf{Чутливість залежить від правила}: Борда є найбільш чутливою до напрямку призначення метрик; РГ демонструє помірні асиметрії зі зворотним напрямком; відносна більшість загалом ближча до симетричної поведінки.
\end{enumerate}

Отримані висновки важливі для теорії демократичних процедур: гетерогенність метрик не слід трактувати лише як «шум» або спотворення. Вона є природним проявом різноманітності виборців і складно взаємодіє з механізмами агрегації, змінюючи підсумкові результати залежно від правила.

\subsection{Теоретичне та прикладне значення результатів}

Результати роботи мають теоретичне й прикладне значення. Теоретично вони уточнюють прояви концентрації міри в контексті просторових моделей голосування, встановлюють зв’язок між вимірністю простору та центральністю виборців і дають інтерпретацію механізмів, через які гетерогенність метрик змінює підсумок. Прикладно ці висновки можуть бути корисними під час проєктування та аналізу виборчих процедур, коли важливо враховувати різні способи «вимірювання» політичної відстані різними групами виборців.

\subsection{Достовірність результатів}

Достовірність результатів забезпечується такими чинниками:
\begin{itemize}
    \item \textbf{Математична строгість}: теоретичні твердження обґрунтовано з використанням закону великих чисел і стандартних фактів теорії збіжності.
    \item \textbf{Статистична надійність}: емпіричні висновки базуються на симуляціях Монте-Карло з $N=200$ незалежними електоральними профілями для кожної конфігурації.
    \item \textbf{Відтворюваність}: код і дані доступні для перевірки та повторення експериментів.
    \item \textbf{Систематичність}: аналіз виконано для різних правил голосування, пар метрик та вимірів простору, що забезпечує повноту порівняння.
\end{itemize}

\subsection{Практичні рекомендації}

На основі отриманих результатів можна сформулювати такі практичні рекомендації:

\begin{enumerate}
    \item \textbf{Для систем на основі методу Борда}:
          \begin{itemize}
              \item Враховувати високу чутливість до призначення метрик: конфігурація «косинус-центр» може давати 60\% розбіжності проти 8.5\% для «L2-центр».
              \item За використання гетерогенних метрик окремо аналізувати, які групи виборців отримують які метрики, оскільки напрямок призначення суттєво змінює результат.
              \item Пам’ятати, що для Борда часто зростає частка сильних розбіжностей (тобто результатів, не відтворюваних жодною базовою лінією).
          \end{itemize}

    \item \textbf{Для низьковимірних просторів ($d \leq 3$)}:
          \begin{itemize}
              \item Очікувати більш виражених ефектів гетерогенності та сильнішої залежності від вибору правила.
              \item Для пар Косинусна--Мінковський можливі напрямкові асиметрії понад 50 п.п. у Борда.
              \item Якщо потрібна більш симетрична поведінка, розглянути поєднання L1/L2 (15--22\% розбіжності з меншими напрямковими ефектами).
          \end{itemize}

    \item \textbf{Для високовимірних просторів ($d \geq 5$)}:
          \begin{itemize}
              \item Ефекти гетерогенності не зникають (8--18\% розбіжності), але не обов’язково змінюються монотонно з ростом $d$.
              \item Вибір метрики лишається важливим: зростання вимірності саме по собі не «скасовує» гетерогенність.
              \item Вибір правила агрегації залишається критичним навіть у високих вимірах.
          \end{itemize}

    \item \textbf{Для аналізу з використанням декомпозиції}:
          \begin{itemize}
              \item Відстежувати співвідношення сильної та периферійно-вирівняної розбіжностей, щоб розуміти \textit{як саме} проявляється вплив гетерогенності.
              \item Висока загальна розбіжність за малої сильної компоненти зазвичай означає «вирівнювання» до однієї з базових ліній, а не появу нових переможців.
              \item Використовувати декомпозицію як діагностику: чи дає гетерогенність справді нові результати, чи лише підсилює одну з базових ліній.
          \end{itemize}
\end{enumerate}

\subsection{Напрямки майбутніх досліджень}

Робота також відкриває кілька напрямків для подальших досліджень:

\begin{enumerate}
    \item \textbf{Теоретичне пояснення взаємодії Борда--Косинусна}: чому «косинус-центр» у Борда дає 7-кратну вищу розбіжність, ніж «L2-центр», і чому 80.8\% цієї розбіжності припадає на периферійно-вирівняну компоненту, а не на появу нових результатів? Чи можна отримати аналітичні межі для цієї асиметрії та пояснити механізм, через який косинусна відстань підсилює власні переваги під агрегацією Борда?

    \item \textbf{Декомпозиція у вищих вимірах}: як змінюється співвідношення сильної та периферійно-вирівняної компонент при переході до більших $d$ (зокрема, понад 10)? Чи є у цих пропорцій гранична поведінка?

    \item \textbf{Нерівномірні геометрії}: як поляризовані або кластеризовані розподіли виборців змінюють патерни декомпозиції та напрямкові асиметрії?

    \item \textbf{Альтернативні функції корисності}: чи змінюються патерни взаємодії метрик і правил голосування за гаусових або квадратичних корисностей?

    \item \textbf{Стратегічне голосування}: як метрична гетерогенність взаємодіє зі стратегічною поведінкою і чи можуть виборці «експлуатувати» структуру декомпозиції?

    \item \textbf{Емпірична валідація}: чи можна оцінити гетерогенність метрик із реальних електоральних даних і чи узгоджуються спостережувані патерни з передбаченнями декомпозиції?

    \item \textbf{Інші правила голосування}: як схвальне голосування, STAR голосування та інші методи взаємодіють із метричною гетерогенністю та описаною декомпозицією?
\end{enumerate}

\section*{Подяки}

Дослідження виконано з використанням обчислювальних симуляцій, реалізованих у Python. Код і дані доступні для перевірки та відтворення результатів.

\newpage

% --- СПИСОК ВИКОРИСТАНИХ ДЖЕРЕЛ ---
\section*{СПИСОК ВИКОРИСТАНИХ ДЖЕРЕЛ}
\addcontentsline{toc}{section}{СПИСОК ВИКОРИСТАНИХ ДЖЕРЕЛ}
\begin{thebibliography}{12}

\bibitem{arrow1963}
Arrow K. J. Social Choice and Individual Values. 2nd ed. New York : Wiley, 1963.

\bibitem{balinski2011}
Balinski M., Laraki R. Majority Judgment: Measuring, Ranking, and Electing. Cambridge, MA : MIT Press, 2011.

\bibitem{black1958}
Black D. The Theory of Committees and Elections. Cambridge : Cambridge University Press, 1958.

\bibitem{condorcet1785}
Condorcet M. Essai sur l’application de l’analyse à la probabilité des décisions rendues à la pluralité des voix. Paris : Imprimerie Royale, 1785.

\bibitem{downs1957}
Downs A. An Economic Theory of Democracy. New York : Harper \& Row, 1957.

\bibitem{enelow1984}
Enelow J. M., Hinich M. J. The Spatial Theory of Voting: An Introduction. Cambridge : Cambridge University Press, 1984.

\bibitem{procaccia2010}
Faliszewski P., Procaccia A. D. AI's War on Manipulation: Are We Winning? // AI Magazine. 2010. Vol. 31, No. 4. P. 53–64.

\bibitem{gibbard1973}
Gibbard A. Manipulation of Voting Schemes: A General Result // Econometrica. 1973. Vol. 41, No. 4. P. 587–601.

\bibitem{ledoux2001}
Ledoux M. The Concentration of Measure Phenomenon. Providence : American Mathematical Society, 2001.

\bibitem{mckelvey1976}
McKelvey R. D. Intransitivities in multidimensional voting models and some implications for agenda control // Journal of Economic Theory. 1976. Vol. 12. P. 472–482.

\bibitem{saari1995}
Saari D. G. Basic Geometry of Voting. Berlin : Springer-Verlag, 1995.

\bibitem{saari2001}
Saari D. G. Chaotic Elections! A Mathematician Looks at Voting. Providence : American Mathematical Society, 2001.

\end{thebibliography}

\end{document}
