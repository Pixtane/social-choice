\documentclass[14pt]{article}
\usepackage[utf8]{inputenc}
\usepackage[ukrainian]{babel}
\usepackage[T2A]{fontenc}
\usepackage{geometry}
\usepackage{setspace}
\usepackage{amsmath}
\usepackage{amssymb}

\geometry{a4paper, left=20mm, top=20mm, bottom=20mm, right=10mm}
\onehalfspacing

\begin{document}

% --- ТИТУЛЬНИЙ АРКУШ ---
\begin{titlepage}
    \centering
    \small
    МІНІСТЕРСТВО ОСВІТИ І НАУКИ УКРАЇНИ \\
    ДЕПАРТАМЕНТ ОСВІТИ І НАУКИ ВИКОНАВЧОГО ОРГАНУ КИЇВСЬКОЇ МІСЬКОЇ РАДИ \\
    (КИЇВСЬКОЇ МІСЬКОЇ ДЕРЖАВНОЇ АДМІНІСТРАЦІЇ) \\
    КИЇВСЬКЕ ТЕРИТОРІАЛЬНЕ ВІДДІЛЕННЯ МАЛОЇ АКАДЕМІЇ НАУК УКРАЇНИ \\
    (КИЇВСЬКА МАЛА АКАДЕМІЯ НАУК) \par
    \vspace{1cm}
    \normalsize
    Відділення: Математика \\
    Секція: Математичне моделювання \par
    \vspace{3cm}
    \Large
    \textbf{МАТЕМАТИЧНЕ МОДЕЛЮВАННЯ ТА СИМУЛЯЦІЙНИЙ АНАЛІЗ ВЛАСТИВОСТЕЙ ГЕТЕРОГЕННОСТІ МЕТРИК У БАГАТОВИМІРНОМУ ІДЕОЛОГІЧНОМУ ПРОСТОРІ} \par
    \vspace{3cm}
    \normalsize
    \flushright
    \textbf{Роботу виконав:} \\
    Прізвище Ім'я По батькові \\
    учень/учениця \_\_\_ класу \\
    Назва навчального закладу \\
    Район \par
    \vspace{0.5cm}
    \textbf{Науковий керівник:} \\
    Прізвище Ім'я По батькові \\
    Посада, науковий ступінь \par
    \vspace{0.5cm}
    \textit{Рекомендую до захисту} \\
    \_\_\_\_\_\_\_\_\_\_ (підпис) \par
    \vfill
    \centering
    КИЇВ -- 2024
\end{titlepage}

\newpage

% --- АНОТАЦІЯ ---
\begin{center}
    \large \textbf{АНОТАЦІЯ}
\end{center}

\noindent \textbf{Назва роботи:} Математичне моделювання та симуляційний аналіз властивостей гетерогенності метрик у багатовимірному ідеологічному просторі. \\
\textbf{Автор:} [Прізвище Ім'я По батькові] \\
\textbf{Відділення:} Математика \\
\textbf{Секція:} Математичне моделювання \\
\textbf{Навчальний заклад:} [Назва навчального закладу] \\
\textbf{Науковий керівник:} [Прізвище Ім'я По батькові, посада] \par

\vspace{0.5cm}

Дана науково-дослідницька робота присвячена комплексному математичному та обчислювальному аналізу гетерогенності метрик відстані у багатовимірних просторових моделях голосування [4]. **Актуальність дослідження** зумовлена необхідністю розуміння того, як ідеологічна складність суспільства та різні способи сприйняття політичних дистанцій виборцями впливають на стабільність демократичних інститутів [6, 7].

У роботі вперше проведено порівняльний аналіз метрик Мінковського ($L1, L2, L\infty$) та косинусної відстані в контексті теорії концентрації міри [8]. **Наукова новизна** дослідження полягає у виявленні феномену «зворотної конвергенції» для системи миттєвого другого туру (IRV): на відміну від теоретичних очікувань, ефективність задоволеності виборців (VSE) для IRV при малих групах кандидатів стрімко падає з 0,71 у 1D до 0,52 у 3D через посилення ефекту «стискання центру» [за результатами симуляцій, 649].

**Метою роботи** є визначення математичних механізмів, через які гетерогенність сприйняття дистанції впливає на результати виборів за правилами відносної більшості, Борда та IRV [8]. **Основними завданнями** було: розробити систему симуляцій Монте-Карло для просторів 1–10 вимірів; проаналізувати швидкість збіжності центральності виборців до теоретичного значення $\sqrt{1/3}$; декомпозувати розбіжності результатів голосування на складники «нових рівноваг» та «ампліфікації крайніх поглядів» [4, 9, 10].

За результатами дослідження встановлено, що гетерогенність метрик призводить до розбіжностей у результатах до 60\% (для методу Борда), а ідеологічна багатовимірність діє як пастка для компромісних кандидатів у мажоритарних системах.

\vspace{0.5cm}
\noindent \textbf{Ключові слова:} соціальний вибір, просторове моделювання, гетерогенність метрик, VSE, ефект стискання центру, концентрація міри, IRV.

\end{document}